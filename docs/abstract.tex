\begin{abstract}
    \textbf{Aim}
    Spatial and temporal patterns of burnt area are controlled by
        availability of fuel,
        fuel moisture,
        natural and anthropogenic ignitions,
        and anthropagenic suppression.
    Here, we map the limitation and sensitivity of burnt area to each of these controls.

    \textbf{Location}
    Global

    \textbf{Methods}
    We describe a simple framework whereby limitations are imposed by:
        fuel discontinuity;
        fuel moisture and atmospheric drying potential;
        lightning and human ignitions;
        and land use.
    Limitations are described from remote sensed and meteorological observations and optimized against Global Fire Emissions Database (GFED4s) burnt area observations.

    \textbf{Results}
    Fuel moisture is shown to be the main limitation of fire over much of the world, (44\% annual average and 36\% during local dry seasons), particularly in the humid forests and cold, slow drying boreal areas.
    Fuel discontinuity is the next limitation (25\% annually and 23\% in the dry season), especially in deserts and dry season grasslands.
    This is followed by land use change (18\% annually, 21\% dry season)
    and then ignitions (13\% annually, 19\% dry season), which is only a significant limiting factor in dry season savanna,
    where rapid drying of fuel built up during the wet season removes all other natural limitations. In these areas, changes in burnt area are actually more sensitive to other controls, typically land use.
    \begin{shaded}
    \textbf{Main conclusions}
    This study contradicts the way basic processes are represented in many global fire models.
    As ignitions only impact burnt area over a limited geographic extent, better representation of controls imposed by fuel loads and moisture is vital. Human ignitions only contribute to a small increase in global burnt area (2%), which is
    offset by the dramatic impact of suppression through anthropogenic land cover changes.
    The assumption
    that humans cause burnt area over much of the world is therefore clearly incorrect, and adequate simulation
    of suppression through land use should become a priority. This result also has implications when considering
    ecosystem services of agricultural land and fire management policies
\end{shaded}
\end{abstract}
