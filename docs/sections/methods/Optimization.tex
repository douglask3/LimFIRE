\subsection{Optimization}

\begin{shaded}
    The framework is optimised against GFED4s observations \citep{Giglio2013}  using normalised least squares in R \citep{rstats}.  This is likely to change, though, so I won't describe it anymore for the moment. GFED4s was re-gridded from 0.25 to a 0.5$^{\circ}$ resolution using ``resample'' in the raster r-package \citep{rraster}.
\end{shaded}

\begin{figure}[!ht]
  \centering
    \includegraphics[width=0.67\textwidth]{../figs/ind_limiataionsaa.pdf}
  \caption{Annual average limitation of each control.}
  \label{fig:Annual_average_con}
\end{figure}

\begin{figure}[!ht]
  \centering
    \includegraphics[width=0.67\textwidth]{../figs/ind_limiataionsfs.pdf}
  \caption{Fire season average limitiation of each control.}
  \label{fig:Season_con}
\end{figure}

Optimization is performed on equation ~\ref{equ:LimFIRE} to ~\ref{equ:LimFIRE.x}, 4, 15, 16 and 18 % un-hardcode equations numbers
(figure ~\ref{fig:lim_lines}). Each control optimizes two parameters associated with it's maximum expected burnt area (equations ~\ref{equ:fx} and ~\ref{equ:LimFIRE.x}, see figure ~\ref{fig:Logistic_fun}).
Parameters relating different fuel moisture, ignition and suppression sources are also optimized (see table ~\ref{tab:optimize})
