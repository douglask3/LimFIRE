\begin{figure}[!ht]
\colorlet{shadecolor}{Mycolor3}
\begin{shaded}
  \centering
    \includegraphics[width=0.9\textwidth]{../figs/gfedComparison.png}
  \caption{Benchmark comparisons against GFED4s \citep{Giglio2013}.}
  \label{fig:benchmark}
\end{shaded}
\colorlet{shadecolor}{Mycolor2}
\end{figure}

\subsection{Benchmarking}
\begin{shaded}
To assess performance of reconstructed burnt area, we'll probably use NME to compare reconstructed fire from the framework with GFED (figure ~\ref{fig:benchmark}) for annual average, seasonal and inter-annual comparisons \citep{kelley2013comprehensive} as recommended by fireMIP \citet{gmd-2016-237, hantson2016status}, and maybe McFaddens $R^{2}$ as per \citep{bistinas2014causal}. Annual average NME comparison comes out at 0.46, which is better than benchmarking null models described in \citet{kelley2013comprehensive}, and outperforms coupled vegetation-fire models contributing to fireMIP \citep{hantson2016status}, although this is to be expected as the framework is driven by, and optimised to, observations \citep{kelley2013comprehensive}.

\end{shaded}
