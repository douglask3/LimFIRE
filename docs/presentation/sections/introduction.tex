\section{Introduction}
\pgfdeclareimage[width=1.0\paperwidth]{header-image}{header_images/kata_tjuta}

\begin{frame}
    \frametitle{It burns where it rains}
    \framesubtitle{Unimodal relationship with moisture}

    \foreach \x in {1, 2, 3, 4, 5} {
        \only<\x> {
        \begin{textblock*}{10cm}(0cm,1.5cm)
            \includegraphics[width=13.0cm]{images/unimodal/p\x.png}%images/unimodal/p\x.png}
    \end{textblock*}
    }}
    %Make clear we are talking about burnt area
\end{frame}

\begin{frame}[label = intro]
    \frametitle{What else controls fire?}
    \framesubtitle{Is it Ignitions? Is it people?}
	\begin{itemize}
		\visible<2-> {\item Fire-vegetation models incorporate ignition schemes}
		\visible<3-> {\item Schemes often include human and lightning caused ignitions}
		\visible<4-> {\item All models with human-caused ignitions show increased burnt area with people}
		\visible<5-> {\item Little effort placed in anthropogenic fire suppression}
	\end{itemize}

\end{frame}


%\begin{frame}
%    \frametitle{What else controls fire?}
%    \framesubtitle{Fire-limitation framework}
%	\begin{itemize}
%		\visible<2-> {\item Map the limitation and sensitivity of burnt area to}
%        \begin{itemize}
%            \visible<3-> {\item Fuel discontinuity}
%            \visible<4-> {\item Fuel moisture and atmospheric drying potential}
%            \visible<5-> {\item lightning and human ignitions}
%            \visible<6-> {\item land use and human suppression}
%        \end{itemize}00
%		\visible<7-> {\item Controls are described from remote sensed and meteorological observations}
%		\visible<8-> {\item optimized againstburnt area observations}00
%	\end{itemize}
%\end{frame}
