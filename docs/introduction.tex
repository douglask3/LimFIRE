\section{Lit Review}

The geographic distribution of are is driven by complex interactions between climate and vegetation.  Fire is also inuenced by human activities, which in turna are  inuenced  by  and  also  impact  on  climate  and  vegetation  properties.  \citep{hantson2016status, kelley2014modelling}.

\citep{hantson2016status}
Coupled vegetation-fire models aim to decribe fire as a function of fuel, moisture and igntions
- Humans affact fire in 3 ways:
	1. increased igntions with population and pasture (as incoporated in e.g. SPITFIRE and LmFire)
	1. Fragmentation factor, from crop and urban cover

\subsection{moisture}
Kauman and Uhl, 1990; Viegas et al., 1999; Drobyshev et al., 2012:
Warmer temperatures are associated with increased burning through increasing vegetation productivity (and hence fuel production) as well as through creating
climate conditions that promote burning.

Moisture of extinction

Veblen and Kitzberger, 2002; Archibald et al., 2009;  Lehmann  et  al.,  2011;  Bradstock  et  al.,  2012
The influence of precipitation is more complex:

Increases in precipitation events increases wetness of fuel and supresses fire in humid regions.
However, in dry regions fuel is a limiting factor for fire spread.  Increases in precipitation, in space or time, will therefore lead to less fire under wetter conditions but more fire at the arid end of the climate gradient.


\subsection{fuel}
(Kerby et al., 2007; Viedma et al., 2009a; Finney et al., 2010)
Arid  conditions  are  also  likely to limit the size of fire,  as fuel loads typically become discontinuous in dry environments, when vegetation is clumped due to e.g.  grazing, or in topographically complex locations.

\citep{thonicke2001role}
Fuel continuity threshold - below 200 gC/m2, fire reduces to zero

\subsection{igntions}

\subsubsection{Lightning}

\subsubsection{Human}
Bradstock et al., 1998; Vazquez and
Moreno, 1998; Chuvieco, 1999; Guyette et al., 2002; Archibald et al., 2009; Padilla and
Vega-Garca, 2011; Price and Bradstock, 2013; Penman et al., 2013
Anthropogenic ignitions may be accidental, deliberately set
for agricultural purposes or for fire management.

Lavorel et al., 2006; Archibald et al., 2009
Recent studies have suggested that climate, vegetation properties and human activities can have different effects on different aspects of the fire regime:      the timing of the fire season
    the prevalent fire type (crown versus ground fires),
    the number of fires,
    and the area burnt.

Chuvieco et al., 2008; Archibald et al., 2009
Recent studies have, in general, shown humans have a more noticeable impact on the number and timing of fires than on the type of fire or the area burnt

\citet{bistinas2013relationships}
used weighted regression to explore the correlation between population density and burnt area. This relationship is unimodal: burnt area initially increases with population density and then decreases. \citet{bistinas2013relationships} showed that the location of peak burnt area varied somewhat on different continents and with different types of land use.

\subsection{supressions}
Syphard  and  Radelo,  2007;  Archibald  et  al.,  2009
The influence of land use operates through two processes:  removal of fuel through crop harvesting or forestry, and fragmentation of natural vegetation which affects the rate of fire spread.

\subsection{Combining Controls}
\hlb{\textbf{More or less lifted staight from thesis}}
 \citet{van2008climate} compared inter-annual variability in burnt area across the tropics and climate variables related to either fuel accumulation (rainfall in the growing season) or fuel drying (rainfall during the fire season). They showed that increased fire was either correlated with fuel accumulation and anti-correlated with dry season rainfall or vice versa.
This suggests that the unimodal relationships of burnt area with factors such as P--E or NPP may be emergent system properties. Thus, in drier areas (which will also have low NPP) fuel availability is the factor limiting the amount of fire; in these regions increasing precipitation leads to increased NPP, increased fuel and hence increased fire. In wetter areas (which will also have higher NPP), fuel is abundant but burning can be limited by fuel wetness; in such areas, increases in rain will further decrease the amount of burning whereas decreases in rain would increase the amount of burning.



\citet{aldersley2011global}
used a regression-tree and random-forest approach to examine the influence of climate, vegetation and human impact on monthly burnt area.
They found that climate and vegetation properties were the most important controls on burnt area: for e.g. fires occurred almost exclusively in months with temperatures $>$ 28$^{\circ}$C and the highest burnt areas occurred at precipitation levels between 350-1100mm.
Whereas cumulative precipitation and lightning were important variables in determining burnt area, variables related to human impacts were generally unimportant.
\hlb{Gross domestic product (GDP) was the most significant of the human impact variables, but was monotonically and negatively correlated with burnt area.}
The regression-tree approach initially uses single variable comparisons to construct the branching structure. Thus, while it allows combinations of variables to be considered together, it only partially deals with co-correlation between these variables.
Furthermore, the use of variables that display unimodal relationships with burnt area (e.g. Mean Annual Precipitation --- MAP) strongly suggests that these variables are surrogates for the actual controls.

\citet{krawchuk2009global} and \citet{moritz2012climate} used Generalized Additive Models (GAMs) to explore relationships between burnt area and 17 climate variables, NPP and two measures of human impact.
Different subsets of the variables were found to be important in different GAMs, but overall NPP (used as a measure of fuel availability) was the most important variable in determining the amount of burning, with variables controlling fuel moisture (in particularly seasonal temperature variables) being the next most important.
\citet{moritz2012climate} further demonstrated that the relative importance of specific controls varied geographically and with biome. In the tropics and warm-temperate regions, NPP was the strongest control on the amount of burning in desert, temperate grassland, temperate savanna and Mediterranean ecosystems, whereas factors influencing fuel drying, specifically dry season precipitation and temperature seasonality, were the strongest controls on the amount of burning in tropical and subtropical dry/moist forests.

\citet{knorr2014impact} optimized a non-linear statistical model of fire focusing on the potential human influences on burnt area by using a set of pre-defined but parameterized relationships describing the important natural controls. Thus, they described the influence of fuel production using a positive monotonic relationship between fraction of Absorbed Photosynthetic Radiation (fAPAR) and burnt area, and the influence of fuel dryness using a positive monotonic relationship between the Nesterov Index (NI) and burnt area.
They then tested relationships between population density amongst different land cover types/socio-economic development regions and fire frequency (roughly the inverse of burnt area).
Using non-linear parameter optimization, they showed that increases in human population result in a significant exponential decrease in fire in all but the most sparsely populated ($<$0.1 people/km\textsuperscript{2}) areas.

\citet{bistinas2014causal} used Generalized Linear Modelling (GLM) to examine the relationships between 11 variables representing vegetation, land use, climate and potential ignition rates (tree cover, grass/shrub cover, NPP, number of dry days, diurnal temperature range, maximum monthly temperature, the ratio of actual to equilibrium evapotranspiration $\alpha$, lightning number, crop area, grazing land area, population density).
The choice of environmental predictor variables was guided by explicit hypotheses about the potential controls of burnt area, and the GLM approach was adopted so as to be able to take account of potential interactions or co-variations between the controls in order to identify the underlying relationships.
\citet{bistinas2014causal} showed that burnt area increases with NPP, number of dry days, maximum monthly temperature, grazing-land area, grass/shrub cover and diurnal temperature range, and decreases with $\alpha$, cropland area and population density.
They further showed that there is no significant relationship with the number of lightning strikes or with tree cover.
Fuel production (NPP) is the most important determinant of burnt area, with factors affecting the rate of fuel curing (e.g. $\alpha$) and fuel dryness (diurnal temperature range) and fire risk (number of dry days, maximum monthly temperature) next in importance, along with factors that influence fuel type (grass/shrub cover).
The simple monotonic relationships between these predictor variables and burnt area are nevertheless sufficient to give rise to complex behavior.
Specifically, \citet{bistinas2014causal} show that the unimodal relationships that have been shown between e.g. mean annual temperature, mean annual precipitation, population density and gross domestic product are secondary consequences of correlations among predictor variables. Thus, the unimodal relationship between population density and burnt areas results from the co-variance of population with production and moisture: arid conditions, where fire is limited by productivity and fuel availability, typically support only low population densities.

Thus, a consensus is emerging from these global analyses about the importance of specific controls on fire. All of the studies show that vegetation productivity is the most important control on burnt area, closely followed by factors that influence fuel drying or curing, but with the relative importance of each depending on local environmental conditions. Ignitions, whether natural or anthropogenic appear to be non-limiting to burnt area --- effectively, there are always enough potential fire starts and fire spread is therefore determined by other controls. As demonstrated very clearly by \citet{Knorr2013} and \citet{Bistinas}, the most significant human impact on fire is through suppression with burnt area decreasing with population density.


Spatial and temporal patterns of burnt area are controlled by:
    availability of fuel;
    fuel moisture;
    natural and anthropogenic ignitions;
    and anthropagenic suppression.
Here, we map the limitation and sensitivity of burnt area to each of these controls.
