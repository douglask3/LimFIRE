\begin{figure}[!ht]
  \centering
    \includegraphics[width=0.67\textwidth]{Model_schematic.pdf}
  \caption{Model description.}
\end{figure}


\section{Methods}

The framework calculates monthly fire from limitations imposed by fuel, moisture, ignition, and anthropagenic supression controls on a 0.5 degree CRUTS3.22 grid \citep{harris2014cru}, between Jan 2000 and Dec 2010 (the period of data overlap). Parameters relating these limiations to fire are optimized against GFED4s remote sensed burnt area observations \citep{Giglio2013}.
%%%%%%%%%%%%%%%%%%%%%%%%%%%%%%%%%%%%%%%%%%%%%%%%%%%%%%%%%%%%%%%%%%%%%%%%%%%%%%%%%%%%%%%%%
%% Overview                                                                            %%
%%%%%%%%%%%%%%%%%%%%%%%%%%%%%%%%%%%%%%%%%%%%%%%%%%%%%%%%%%%%%%%%%%%%%%%%%%%%%%%%%%%%%%%%%

\subsection{Overview}
The framework assumes that 100\% burnt area occurs in perfect fire conditions,  i.e. complete fuel coverage, no moisture, saturated igntions and no agricultural or urban fragmentation. This is analogous to areas in Northern Australia and parts of the sahel which experiance buring each year.
Burnt area is reduced as each limitation becomes sub-optimal, i.e.
    fuel loads become discontinuous  (e.g. desert areas)
    or too moist (e.g. Humid evergreen forests),
    if their is lack of ignition \citep[shown to be an influence inter annual variability in parts Southern Australia][](bradstock2010biogeographic),
    or with increased human influance on the landscape (e.g. cropland or urban areas).
Fraction burnt area ($F$) is therefore the product of the maximum allowed burnt area for each limitation ($F_i$)
\begin{equation}
    F=\Pi_{i} F_i
\end{equation}

This maxium burnt area is related to fuel loads, moisture, ignitions and supression via the logistic function, as per \citet{bistinas2014causal} (see figure ~\ref{Logistic_fun_fig}):

\begin{equation}
    f(x) = 1 / (1 + e^{-k \cdot (x - x_0)})
\end{equation}
where $k$ descibed the steepness of the curve and $x_0$ is the curves midpoint.

\begin{figure}[!ht]
  \centering
    \includegraphics[width=0.67\textwidth]{Logistic_fun.pdf}
  \caption{Logistic function.}
  \label{Logistic_fun_fig}
\end{figure}

As fire increases with increasing fuel load ($F_w$) and igntions ($F_{ig}$), and decreases with moisture ($F_{\omega}$) and anthropagenic supression ($F_s$), therefore:

\begin{equation}
    \begin{split}
        F_{w} = f(w) \\
        F_{\omega} = 1 - f(\omega) \\
        F_{ig} = f(ig) \\
        F_{s} = 1- f(s)
    \end{split}
\end{equation}



%%%%%%%%%%%%%%%%%%%%%%%%%%%%%%%%%%%%%%%%%%%%%%%%%%%%%%%%%%%%%%%%%%%%%%%%%%%%%%%%%%%%%%%%%
%% Limitations                                                                         %%
%%%%%%%%%%%%%%%%%%%%%%%%%%%%%%%%%%%%%%%%%%%%%%%%%%%%%%%%%%%%%%%%%%%%%%%%%%%%%%%%%%%%%%%%%
\subsection{Inputs}

\subsubsection{Fuel ($w$)}
There are two options for us to choose from:
\begin{enumerate}
    \item \textit{NPP product} as a proxy for total fuel load. Thi is the option I'm using at the moment. The product comes from Terra/MODIS Net Primary Production Yearly L4 Global 1 km MOD17A3 \citep{nasa2012terra}
    \item \textit{A measure of fractional cover} as a proxy for fuel continuity (i.e, annual mean or maximum fAPAR, as used by \citet{knorr2014impact,knorr2016climate} or MODIS fractional cover). \citet{bistinas2014causal} uses SeaWiFs fAPAR, but this product finishes in 2007, so would reduce our comprison period. Other fAPAR/Fractional cover products would probably cover a larger period.
\end{enumerate}

It could be that we use the measure most apprprate for a given application.

\subsubsection{Moisture ($\omega$)}

$\omega$ represents mean fractional water content, and is split between live and dead fuel:

\begin{equation}
    \omega = (live + M \cdot dead) / (1 + M)
\end{equation}

where M is an optimized parameter, representing the importance of dead fuel moisture realtive tom live.

\paragraph{Live Fuel.}
The ratio of actual to potential evaporation \citep[$\alpha$][]{prentice1993simulation}, a measure of available water in relation to plant demand, is used to describe fractional water content of live fuel as per \citet{harrison2010fire, bistinas2014causal}

\paragraph{Dead Fuel} moisture is represented by a simple equilibrium moisture content ($m_{mq}$) calculation from \citep{viney1991review} which combines daily precipitaion ($Pr$) and atmopheric drying potential via temprature ($T$) and relative humidity ($H_r$):

\begin{equation}
     m_{mq, daily})=
        \begin{cases}
            10 - (T - H_r) / 4 ,& \text{if } Pr\leq 3 \text{mm}\\
            100,              & \text{otherwise}
        \end{cases}
\end{equation}

On a monthly timestep, this simplifies to:

\begin{equation}
     m_{mq})=
        (10 - (T - H_r) / 4) \cdot (1 - WD)
        + 100 \cdot WD
\end{equation}
where WD is the fraction of wet days, defined as a day where $Pr > 3mm$.


\subsubsection{Igntions ($ig$)}

Igntions combines lightning igntions ($L_{ig}$), ignitions from pasture and local population.

\begin{equation}
    ig = L_{ig} + P \cdot Pasture + D \cdot Population\text{ }Density
\end{equation}
where P and D are opimized paramters.

But could add background igntions could also be considered

\begin{equation}
     ig = 1 + N \cdot L_{ig} + P \cdot Pasture + D \cdot Population\text{ }Density
\end{equation}
where N, P and D are opimized paramters describing the importance of their respective ignition source relative to background igntions.

\paragraph{Lightning}

Lightning is calculated from flash counts
from the Lightning Imaging Sensor \citep[LIS][http://grip.nsstc.nasa.gov/]{christian1999optical, christian1999lightning}.
This contains both cloud-to-ground (avavailbe for igntions) and inter-cload (not avavailbe for ignitions) strikes.
CG lighting is calculated using \citet{kelley2014improved}:

\begin{equation}
    CG = L * min(1, 0.0408 \cdot FL^{-0.4180}
\end{equation}

\citet{kelley2014improved} also extracted dry-only lighting as available igntion sources. I'm implementing this as the moment, but I'm not to sure if I should:

\begin{equation}
    L_{ig} = 0.8533 \cdot CG \cdot e^{-2.835 \cdot WD}
\end{equation}


\paragraph{Pasture and population density}
\label{Pasture}
Pasture and population density are taken from the HYDE dataset \citep{klein2007mapping}, and are interpolated from a decadle to a monthly timestep.

\subsubsection{Supression ($s$)}

Supression combines urban and cropland area and population desnity

\begin{equation}
    s = urban + C \cdot Crop + H \cdot Population\text{ }Density
\end{equation}

where $C$ and $H$ are optimized parameters.

\textbf{Urban} and \textbf{cropland} areas are taken from HYDE and processed as per pasture and population desnity (section ~\ref{Pasture})


%%%%%%%%%%%%%%%%%%%%%%%%%%%%%%%%%%%%%%%%%%%%%%%%%%%%%%%%%%%%%%%%%%%%%%%%%%%%%%%%%%%%%%%%%
%% Optimization                                                                        %%
%%%%%%%%%%%%%%%%%%%%%%%%%%%%%%%%%%%%%%%%%%%%%%%%%%%%%%%%%%%%%%%%%%%%%%%%%%%%%%%%%%%%%%%%%
\subsection{Optimization}

%%%%%%%%%%%%%%%%%%%%%%%%%%%%%%%%%%%%%%%%%%%%%%%%%%%%%%%%%%%%%%%%%%%%%%%%%%%%%%%%%%%%%%%%%
%% Benchmarking                                                                        %%
%%%%%%%%%%%%%%%%%%%%%%%%%%%%%%%%%%%%%%%%%%%%%%%%%%%%%%%%%%%%%%%%%%%%%%%%%%%%%%%%%%%%%%%%%
\subsection{Benchmarking}


%%%%%%%%%%%%%%%%%%%%%%%%%%%%%%%%%%%%%%%%%%%%%%%%%%%%%%%%%%%%%%%%%%%%%%%%%%%%%%%%%%%%%%%%%
%% Analysis                                                                            %%
%%%%%%%%%%%%%%%%%%%%%%%%%%%%%%%%%%%%%%%%%%%%%%%%%%%%%%%%%%%%%%%%%%%%%%%%%%%%%%%%%%%%%%%%%
\subsection{Analysis}


\subsubsection{Limitation}

\begin{equation}
    \bar{l_{i, X}} = \frac{l_{i, X}}{\sum_{j} l_{j, X}}
\end{equation}

\subsubsection{Sensitivity}

\begin{equation}
    \bar{dl_{i, X}} = \frac{dl_{i, X} \cdot \Pi_{j} l_{j, X}}{l_{i, X}}
\end{equation}

where $dl_{i, X}$ is the gradient of $l_{i, X}$ relative to the maximum possible gradient of $l_{i}$, i.e:

\begin{equation}
    dl_{i, X} = \frac{dl_{i, X} / dx}{dl_{i, l = 0.5} / dx}
\end{equation}

where

\begin{equation}
    \frac{dl_{i}}{dx}
\end{equation}
