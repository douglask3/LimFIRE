\section{Fire Limitation Skype-up, 25/10/2016}

\subsection{Framework}

\begin{itemize}
\itemsep1pt\parskip0pt\parsep0pt
\item
  The framework should be flexible and easily accessible so other people can use it for their own projects.
\item
  The initial paper should use free available, global data, so that it could be reproduced as the basis for other people's projects.
\item
  Less accessible data could be introduced in later studies to answer more specific/regional science questions.
\end{itemize}

\subsection{Control Categories}

\subsubsection{Fuel}

\begin{itemize}
\itemsep1pt\parskip0pt\parsep0pt
\item
  Decided it would be better to use a fuel (dis-)continuity proxy (i.e fAPAR, modis fractional cover etc) rather than fuel/production (such as NPP products) as:

\begin{itemize}
\itemsep1pt\parskip0pt\parsep0pt
\item
  Fuel continuity is what the framework should really be describing
\item
  All NPP products we could think of are a bit dubious
\end{itemize}

\end{itemize}



\begin{itemize}
\itemsep1pt\parskip0pt\parsep0pt
\item
  SeaWIFS (used in Yannis' paper) discontinued in 2005, so ideally need to find other products covering longer timespan.
\end{itemize}

\paragraph{Actions}

\begin{itemize}
\itemsep1pt\parskip0pt\parsep0pt
\item
  Yannis to look into other fAPAR/ fractional cover products. These don't need to be on a monthly timestep - annual would be fine cos we may convert sub-annual data to fuel-load coverage with something like a 12-month running mean or max. But monthly/sub-monthly would be better.
\end{itemize}

\subsubsection{Moisture}

\paragraph{Live fuel:}

\begin{itemize}
\itemsep1pt\parskip0pt\parsep0pt
\item
  Uncertainty about how STASH actually works, and if r-stash has all the SPLASH stuff in it.
\end{itemize}

\paragraph{Dead Fuel:}

\begin{itemize}
\itemsep1pt\parskip0pt\parsep0pt
\item
  Question as to whether an atmospheric drying potential index is sensible for dead fuel moisture: Dead fuel is connected to the soil and would be influenced by soil moisture as well.
\item
  Could fuel be aggregated into one measure (such as in GLOBFIRM, SIMFIRE or INFERNO)? Or could both soil-controlled and atmosphere-controlled components be obtained from the ~same source (i.e. STASH)?
\end{itemize}

\paragraph{Actions:}

\begin{itemize}
\itemsep1pt\parskip0pt\parsep0pt
\item
  Chantelle: See if there are fire danger indices shown to be a good measure for fuel moisture outside of their development range (or did we decide this was a no-go already?)
\item
  Doug:

\begin{itemize}
\itemsep1pt\parskip0pt\parsep0pt
\item
  Find out what happens in STASH.
\item
  See what SPLASH uses for elevation and field capacity.
\item
  Look for elevation maps for STASH
\item
  See if STASH could provide some description of dead fuel or fuel as a whole
\item
  I'll also see what SIMFIRE does as well.
\end{itemize}

\end{itemize}



\begin{itemize}
\itemsep1pt\parskip0pt\parsep0pt
\item
  Toby: Dig out a global soil/field capacity map.
\item
  Did someone mention ECMWF top soil moisture product as well?
\end{itemize}

\subsubsection{Lightning}

\begin{itemize}
\itemsep1pt\parskip0pt\parsep0pt
\item
  Don't remove wet lightning - thus would be double-counting moisture
\item
  Might be able to replace LIS data with non-climatological lightning data from met office
\end{itemize}

\paragraph{Actions:}

\begin{itemize}
\itemsep1pt\parskip0pt\parsep0pt
\item
  Chantelle: Look into met office lightning data. Check to see if it id flash count, ground strikes etc. Kerry Smout-Day might have some information (I'll forward you an email from her)
\end{itemize}

\subsubsection{Anthropogenic Suppression}

\begin{itemize}
\itemsep1pt\parskip0pt\parsep0pt
\item
  HYDE is a good product compared to e.g. GRUMP, which has problems with within country aggregation.
\item
  Yannis has info of different pop dens datasets and why HYDE is a good choice that he could dig out at some point if required.
\item
  Probably only need urban or pop density, so will decide on which variable to use when other points below have been addressed.
\item
  It would be nice to test GDP at some point, but probably no good global GDP products.
\item
  Human Footprint index might be useful for either information about specific impacts (i.e, road fragmentation) or general human disturbance.
\item
  There are some road products as well which could help describe fragmentation (and possibly ignitions? Although that might be a can of worms only to be opened if we run out of spaghetti. Or when someone uses the framework for a regional study (i.e. Amazon?))

\end{itemize}

\paragraph{Actions:}

\begin{itemize}
\itemsep1pt\parskip0pt\parsep0pt
\item
  Chantelle: See if there are any new GDP products without administrative area aggregation problems.
\item
  Chantelle: Find out info about CARDOSO road product (although this is probably for future studies rather than now).
\item
  Toby: Info about GROADS product
\item
  Yannis: Look into Human Footprint Index product, inc. what fields are used to construct it and if these fields are available separately.
\end{itemize}

\hfill \break

\textit{Anyone looking into datasets, remember to check to see:}

\begin{itemize}
\itemsep1pt\parskip0pt\parsep0pt
\item
  If they are something we could make a 0.5-degree resolution map from
\item
  That if they change rapidly (i.e lightning), that they are on at least a monthly timestep.
\item
  If that have had some kind of evaluation
\item
  Are freely available
\end{itemize}

\subsection{Optimization}

\begin{itemize}
\itemsep1pt\parskip0pt\parsep0pt
\item
  Rhys suggests using pymc (python package), to perform the following:

\begin{itemize}
\itemsep1pt\parskip0pt\parsep0pt
\item
  Using Bayesian approach
\item
  Requires priors estimates
\item
  Produces posterior distribution of parameters
\item
  Pathway to parameter estimate and distribution of posterior gives idea of significance and confidence of parameter
\item
  Identifies correlation between parameter estimates
\item
  Rhys: does that sound right?
\item
  Does this use metropolis-hastings algorithm?
\end{itemize}
\end{itemize}



\begin{itemize}
\itemsep1pt\parskip0pt\parsep0pt
\item
  It would be good for Toby and Rhys to chat about optimisation at some point.
\item
  Did we decide anything about testing for variable correlation before optimisation?
\end{itemize}

\subsection{Benchmarking}

\begin{itemize}
\itemsep1pt\parskip0pt\parsep0pt
\item
  Benchmarking procedure to be worked out later.
\item
  Question about training vs testing data. I.e, should data be trained on x \% of data points and benchmarked against remaining y points?
\end{itemize}
                    
